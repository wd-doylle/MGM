\documentclass[11pt, oneside]{article}   	% use "amsart" instead of "article" for AMSLaTeX format
\usepackage{geometry}                		% See geometry.pdf to learn the layout options. There are lots.
\geometry{letterpaper}                   		% ... or a4paper or a5paper or ... 
%\geometry{landscape}                		% Activate for rotated page geometry
%\usepackage[parfill]{parskip}    		% Activate to begin paragraphs with an empty line rather than an indent
\usepackage{graphicx}				% Use pdf, png, jpg, or eps§ with pdflatex; use eps in DVI mode
\usepackage{listings}				% TeX will automatically convert eps --> pdf in pdflatex		
\usepackage{amssymb}
\usepackage{amsmath}
\usepackage{xcolor}
\usepackage{float}
%SetFonts

%SetFonts


\title{Brief Article}
\author{The Author}
%\date{}							% Activate to display a given date or no date

\begin{document}
\maketitle
\section{MGM-Floyd}
MGM-Floyd aims to seek an optimal composition path of initial matchings between any two vertices on the supergraph. The algorithm mainly consists of three parts:
\begin{itemize}
\item[affinityScore] Define\\
 \textbf{X} partial permutation matrix indicating the node correspondence\\
 \textbf{K}($\in R^{n_1n_2\times n_1n_2}$) affinity matrix whose diagonal (off-diagonal) encodes the node-to-node affinity (edge-to-edge affinity) between two graphs.\\
 Formally, we have 
 \begin{align} % requires amsmath; align* for no eq. number
    J(X) = \min\limits_{X\in \{0,1\}^{n_1\times n_2}} vec(X)^TKvec(X)
 \end{align}
In code we have
\begin{lstlisting}[language={[ANSI]C},keywordstyle=\color{blue!70},commentstyle=\color{red!50!green!50!blue!50},frame=shadowbox, rulesepcolor=\color{red!20!green!20!blue!20}] 
vecX = np.reshape(X, (num_graph,num_graph,num_node*num_node,1))
vecXt = vecX.transpose((0, 1, 3, 2))
normScore =
 np.matmul(np.matmul(vecXt, K), vecX).reshape(num_graph, num_graph)/1
\end{lstlisting}

\item[pairwiseConsistency]  Given $\{G_k\}^N_{k=1}$ and matching configuration X\\
 for any pair $G_i$ and $G_j$, formally we have
 \begin{align} % requires amsmath; align* for no eq. number
    C(X_{ij},X) = 1 - \frac{\sum_{k=1}^N||X_{ij}-X_{ik}X_{kj}||_F/2}{nN}
 \end{align} 
In code we have
\begin{lstlisting}[language={[ANSI]C},keywordstyle=\color{blue!70},commentstyle=\color{red!50!green!50!blue!50},frame=shadowbox, rulesepcolor=\color{red!20!green!20!blue!20}] 
Xt = X.transpose((0, 1, 3, 2))
sigma = np.abs(X - np.matmul(X, Xt)).sum((2, 3))
pairwiseConsistency = 1 - sigma / (2 * num_graph * num_node)
\end{lstlisting}

\item With the tools above we can implement the algorithm
\begin{figure}[H]
   \centering
   \includegraphics[scale=0.7]{p1} % requires the graphicx package
\end{figure}
It is obvious that line 8 is only different from line 2~7 of $\lambda$, so we can implement line 2~7 first
\begin{lstlisting}[language={[ANSI]C},keywordstyle=\color{blue!70},commentstyle=\color{red!50!green!50!blue!50},frame=shadowbox, rulesepcolor=\color{red!20!green!20!blue!20}] 
    for i in range(num_graph):
        X_xvy = np.matmul(X[:, i, None], X[i, None,:])
        S_org = affinityScore(X,K,num_graph,num_node)
        S_xvy = (1-lam)*affinityScore(X_xvy, K,num_graph, num_node)
        compare = S_xvy > S_org
        for j in range(24):
            for k in range(24):
                if compare[i,j]:
                    X[j,k] = X_xvy[j,k]
    return X
\end{lstlisting}
Line 3 of code corresponds to line 4 of algorithm above, line 4 corresponds to line 5, line 8~9 corresponds to line 6~7. For line 8 of the algorithm above we just reuse line 2~7. At last we can wrap them into a function.
\begin{lstlisting}[language={[ANSI]C},keywordstyle=\color{blue!70},commentstyle=\color{red!50!green!50!blue!50},frame=shadowbox, rulesepcolor=\color{red!20!green!20!blue!20}] 
def Line2_7(X, K, num_graph, num_node, lam):
    if lam == 0:
        for i in range(num_graph):
            X_xvy = np.matmul(X[:, i, None], X[i, None,:])
            S_org = affinityScore(X,K,num_graph,num_node)
            S_xvy = (1-lam)*affinityScore(X_xvy, K,num_graph, num_node)
            compare = S_xvy > S_org
            for j in range(24):
                for k in range(24):
                    if compare[i,j]:
                        X[j,k] = X_xvy[j,k]
        return X

    else:
        for i in range(num_graph):
            pairwise_consistency = pairwiseConsistency(X,num_graph,num_node)
            X_xvy = np.matmul(X[:, i,None], X[i,None, :])
            S_org = (1-lam)*affinityScore(X,K,num_graph,num_node) 
            			+ lam*pairwise_consistency
            S_xvy = (1-lam)*affinityScore(X_xvy, K,num_graph, num_node) 
            			+ lam*pairwise_consistency
            compare = S_xvy > S_org
            for j in range(24):
                for k in range(24):
                    if compare[j,k]:
                        X[j,k] = X_xvy[j,k]
        return X
\end{lstlisting}
So the function mgm\_floyd is very simple
\begin{lstlisting}[language={[ANSI]C},keywordstyle=\color{blue!70},commentstyle=\color{red!50!green!50!blue!50},frame=shadowbox, rulesepcolor=\color{red!20!green!20!blue!20}] 
def mgm_floyd(X, K, num_graph, num_node):
    X = Line2_7(X, K, num_graph, num_node, 0)
    X = Line2_7(X, K, num_graph, num_node, 0.3)
    return X
\end{lstlisting}
Display results through the test
\begin{figure}[H]
   \centering
   \includegraphics[scale=0.2]{p2} % requires the graphicx package
\end{figure}
\begin{figure}[H]
   \centering
   \includegraphics[scale=0.2]{p3} % requires the graphicx package
\end{figure}

\end{itemize}




\end{document}  